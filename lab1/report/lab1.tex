\documentclass[aps, reprint,amsmath,amssymb]{revtex4-1} %APS Journal
\usepackage[T1]{fontenc}
\usepackage[utf8]{inputenc}
\usepackage{lmodern}
\usepackage{microtype}
\usepackage{graphicx}
\usepackage{siunitx}
\usepackage{bm}

\renewcommand{\vec}[1]{\boldsymbol{#1}}
\newcommand{\mat}[1]{\mathbf{#1}}
\newcommand{\uv}[1]{\vec{\hat{#1}}}
\newcommand{\x}{\vec{\hat{x}}}
\newcommand{\y}{\vec{\hat{y}}}
\newcommand{\z}{\vec{\hat{z}}}
\renewcommand{\d}{\partial}
\renewcommand{\L}{\mathcal{L}}
\renewcommand{\inf}{\infty}

\begin{document}
%----------------------------------------------------------------------
% title
%----------------------------------------------------------------------
\title{PHY64 Experiment 1: The Cavendish Experiment}
\author{Matthew S. E. Peterson}
\author{Jackson Burzynski}
\date{\today} 
\maketitle

%----------------------------------------------------------------------
% Body
%----------------------------------------------------------------------
\section{Introduction}

\section{Theory}

We have a long rod of length $2d$ and width $w$, with two spherical masses
of radius $r$ at each end.  The rod has linear mass density $\sigma$, and
the two masses have mass $m$. Two large tungsten masses, each with mass $M$
and radius $R$, are placed near the smaller masses. The rod is allowed
to rotate about its center, which we will use as the origin, with the rod
and all masses existing in the $x$-$y$ plane. Each small spherical mass has
moment of inertia $I_\text{sphere} = 2mr^2/5 + md^2$, and the beam has
moment of inertia $I_\text{beam} = 2d\sigma(4d^2 + w^2)/12$. Thus, the
total moment of inertia is
\begin{equation}
    \label{eq:moment_of_inertia}
    I = 2I_\text{sphere} + I_\text{beam} 
    = \frac{2}{5} m (2 r^2 + 5 d^2) + \frac{1}{6}d\sigma(w^2 + 4d^2).
\end{equation}
We can then determine the equation of motion of the rod using
\begin{equation}
    \sum_i \vec{\tau}_i = I \frac{d^2\vec{\theta}}{dt^2},
\end{equation}
where we sum over all of the torques acting on the system. 

CoUntitled-1nsider the system with only one of the tungsten spheres. Due to symmetry,
the other sphere will result in the same torque acting on the system, and
so we need only multiply by two at the end. Let $\vec{r} = d\x + b\y$ be the
position of the tungsten sphere. The two smaller spheres are at positions
$\vec{r}_1 = d \cos\theta\,\x + d \sin \theta \, \y$ and $\vec{r}_2 = -d
\cos \theta\,\x - d \sin \theta \, \y$. Thus, the separations between the
tungsten spheres and the smaller spheres are
\[
    \vec{R}_1 = \vec{r} - \vec{r}_1 = d(1 - \cos\theta)\x + (b -
    d\sin\theta) \y
\]
and
\[
    \vec{R}_2 = \vec{r} - \vec{r}_2 = d(1 + \cos\theta)\x + (b + d \sin
    \theta) \y.
\]
The forces acting on each small sphere are
\begin{equation}
    \label{eq:forces}
    \vec{F}_k = \frac{G M m}{R_k^2}\,\uv{R}_k,
\end{equation}
where $k = 1$ or 2 depending on the sphere we look at. The corresponding
torques are therefore
\begin{equation}
    \label{eq:torques}
    \vec{\tau}_k = \vec{r}_k \times \vec{F}_k.
\end{equation}

We now look to determine the torque acting on the rod itself. A small mass
element of the rod has mass $dm = \sigma\,dl$, where $dl$ is an
infinitesimal length. Choose a point $l$ on the rod, with $-d\leq l \leq
d$. Then the position of that point is $\vec{p} = l\cos\theta\,\x +
l\sin\theta\,\y$. The force acting on that particular point on the rod is
\[
    d\vec{F} = \frac{G M \sigma}{(\vec{r} - \vec{p})^2} \frac{\vec{r} -
    \vec{p}}{|\vec{r} - \vec{p}|} \, dl
\]
    
\section{Data}

\section{Analysis}

\section{Error}

\section{Conclusion}


\end{document}
